% This is file JFM2esam.tex
% first release v1.0, 20th October 1996
%       release v1.01, 29th October 1996
%       release v1.1, 25th June 1997
%       release v2.0, 27th July 2004
%       release v3.0, 16th July 2014
%   (based on JFMsampl.tex v1.3 for LaTeX2.09)
% Copyright (C) 1996, 1997, 2014 Cambridge University Press

\documentclass[12pt]{jfm}
\usepackage[a4paper, margin=1.5cm]{geometry}
\usepackage{booktabs}
\usepackage{hyperref}
\hypersetup{
    colorlinks=true,
    linkcolor=blue,     % color of internal links (change box color with linkbordercolor)
    citecolor=blue,     % color of links to bibliography
    filecolor=blue,     % color of file links
    urlcolor=blue       % color of external links
}

\usepackage{graphicx}
\usepackage{epstopdf, epsfig}
\usepackage{natbib}
\newtheorem{lemma}{Lemma}
\newtheorem{corollary}{Corollary}

\shorttitle{Xuyuan Zhang Econ 504}
\shortauthor{Xuyuan Zhang Econ 504}

\title{Econometrics Replication}

\author{Xuyuan Zhang\aff{1}
  \corresp{\email{\href{mailto:zxuyuan@umich.edu}{zxuyuan@umich.edu}}}}

\affiliation{\aff{1}Department of Economics, University of Michigan, Ann Arbor, USA.}

\linespread{1.25}

\begin{document}

\maketitle

\section{Introduction} \label{sec:introduction}

Previous literature has consistently shown that adult education programs play a key role in reducing poverty rates in various regions \citep{RePEc:wbk:wboper:9767, doi:10.1086/590461}, but previous designs suffer from irrelevance to daily work and high dropout rates. Therefore, it is necessary to think of a way to directly help adults gain the benefits of education. This paper proposes an innovative strategy to empower adults by teaching them to use mobile phones to acquire essential skills.

Recent years have witnessed the widespread of mobile phones and the proliferation of mobile phones and their services has played a significant role in shaping economic behaviors.  highlights the substantial influence of mobile phone services on farmer's .


reduction in grain price dispersion, which has been demonstrated by \citep{10.1257/app.2.3.46} highlights that mobile 


 Mobile phone technology 
could also affect returns to education by allowing households to use the technology 
for other purposes, such as obtaining price and labor market information, and facilitating informal private transfers (Aker and Mbiti 2010).

The return of \citep{GONZALEZ2024103228} 

\subsection{Experiment Design} \label{subsec:ExperimentDesign}

Randomized controlled trials (RCT) has been widely used in economic design. This study use
\section{Main Result} \label{sec:mainresult}

\section{}

\begin{table}[htb]
  \begin{center}
    \begin{scriptsize}
    \caption{contamination of the randomization Check}
    \label{tab:contamination}
    \begin{tabular}{lllllll}
\toprule
Variable & Mean without abc & SD without abc & Mean with abc & SD with abc & Diff & std \\
\midrule
Are you the household head? & 0.560 & 0.497 & 0.547 & 0.498 & -0.01 & (0.02) \\
Respondent is Hausa & 0.715 & 0.452 & 0.721 & 0.449 & 0.01 & (0.03) \\
Number of household members & 8.422 & 4.054 & 8.328 & 4.074 & 0.02 & (0.25) \\
Percentage of children under 15 who have some education & 0.279 & 0.276 & 0.269 & 0.270 & -0.00 & (0.02) \\
Number of asset categories owned by household & 4.990 & 1.609 & 4.979 & 1.575 & -0.03 & (0.10) \\
Household experienced drought in past year & 0.385 & 0.487 & 0.380 & 0.486 & -0.03 & (0.03) \\
Household owns a cell phone (excluding group phone) & 0.296 & 0.457 & 0.295 & 0.457 & -0.00 & (0.03) \\
Access to household or village-level cell phone & 0.763 & 0.426 & 0.798 & 0.402 & 0.04* & (0.02) \\
Respondent has used cell phone since last harvest & 0.542 & 0.499 & 0.573 & 0.495 & 0.04 & (0.03) \\
Respondent has made call & 0.691 & 0.463 & 0.725 & 0.447 & 0.03 & (0.04) \\
Respondent has received call & 0.858 & 0.349 & 0.868 & 0.339 & 0.03 & (0.03) \\
\bottomrule
\end{tabular}


    Notes: * significant at the 10 percent level; ** significant at the 5 percent level; *** significant at the 1 percent level.
    \end{scriptsize}
  \end{center}
\end{table}



\newpage

\bibliographystyle{jfm}
% Note the spaces between the initials
\bibliography{jfm-instructions}

\newpage
\appendix

\section{}\label{appA}

This appendix contains the code that is used to replicate the main result of the paper. The Appendix \ref{appB} contains the detailed executed log file in the operation.

\section{}\label{appB}

\end{document}
